\RequirePackage{setspace}
\documentclass{ejb}
\setcounter{page}{1} % Please input the first page number in place of "1". This will change the starting page number of the paper.

%       -------------------------------------------------------------------

\usepackage{comment}
\usepackage{rotating}
\usepackage{xspace}
\usepackage{tikzscale}
\usepackage{filecontents}
\usepackage{tikz}
\usepackage{fancyvrb}
%\usepackage{subcaption}
\usepackage{pgfplotstable}
\usepackage{longtable}
\usepackage{pgfplots}
\usepackage{hyperref}
\usepackage{helvet}
\usepackage[eulergreek]{sansmath}
\usepackage{graphicx}
\usepackage{subfig}
\usepackage{pdflscape}
\usepackage{multirow}
\usepackage{fancyhdr}
\usepackage{pdflscape}
%\pagestyle{fancy}
%\chead{\textit{December 2023\\Draft: Do not cite or quote without permission.}}
%\cfoot{\textit{Draft: Do not cite or quote without permission.}}
%\renewcommand{\headrulewidth}{0.4pt}

\pgfplotsset{compat=1.3}

\fvset{formatcom=\color{black},fontsize=\scriptsize,frame=single,rulecolor=\color{black},baselinestretch=1,framesep=5mm}
\renewcommand{\FancyVerbFormatLine}[1]{\textcolor{blue}{#1}}
\DefineVerbatimEnvironment {GAMSCode*}{Verbatim}
{framesep=5mm,frame=none,fontsize=\scriptsize, baselinestretch=1}

\newcommand{\cv}{\mbox{\boldmath$c$}}
\newcommand{\GAMS}{\textsc{gams}\xspace}
%       -------------------------------------------------------------------

\graphicspath{{.}}

\begin{document}

\title{International trade in agricultural products at the
{U.S.} state level
%\footnote{
%		The findings and conclusions in this paper are
%	those of the authors and should not be construed to represent any
%	official USDA or U.S. Government determination or policy. This
%	research was supported in part by a cooperative agreement
%	(58-3000-0-0035) between the U.S. Department of Agriculture,
%	Economic Research Service, and the University of Nebraska at
%	Lincoln.}
	}            
%\author{\small
%Edward J. Balistreri\footnote{
%University of Nebraska-Lincoln. (email:
%{\texttt edward.balistreri@unl.edu}).}\\
%Thomas F. Rutherford\footnote{
%University of Wisconsin.  (email:
%rutherford@aae.wisc.edu).} and\\
%Steven S. Zahniser\footnote{
%Economic Research Service, USDA.  (email:
%steven.zahniser@usda.gov).}
%}


%
\maketitle

\vspace{-30pt}
\begin{jelcodes}
F10, C81
\end{jelcodes}
\vspace{-10pt}
\begin{keywords}
State Trade; Agriculture Exports; Agriculture Imports; State-level
Supply Chains
\end{keywords}
\vspace{-10pt}
\section*{Extended Abstract:}
\vspace{-5pt}
In this paper we develop and apply a new method for estimating
state-level international trade in agricultural products.  The method
is based on a fundamental proposition from trade theory, which posits
equal absorption shares of regionally differentiated goods in the
absence of trade frictions.  Adopting this as an assumption allows us
to calibrate a commodity-specific Armington demand system to
input-output accounts that establish supply and demand vectors for
each state and each port. We then utilize the structure with trade
frictions included to establish the benchmark flows of goods from the
point of production to domestic uses and port-level exports.
Similarly, we distribute port-level imports to their use in specific
states.  While our primary focus is on international trade the method
yields a bilateral interstate trade matrix for each commodity.

Our work complements recent advancements in open-source calculations
of state input-output accounts by the Wisconsin National Data
Consortium (WiNDC).\footnote{See \url{https://windc.wisc.edu/}.}
The WiNDC project focuses on publicly available data sources and a
series of routines that generate micro-consistent subnational
accounts.  Previously available subnational accounts where both
expensive and proprietary, which made them less ideal for
research.\footnote{The IMPLAN (\url{https://implan.com/}) state
	accounts are an example of a closed-source commercial
	alternative.}
The academic based WiNDC project has many advantages including its
accessibility, transparency, and extensibility to particular research
questions.  

Representing bilateral interstate and state-level international trade
in the WiNDC accounts remains as a challenge. A lack of reliable data 
on interstate trade favors a \emph{pooled} national market
formulation, which ultimately limits the structural options for
researchers.  At first the U.S. Department of Commerce's Bureau of
Census (Census) reports in the Commodity Flow Survey (CFS) might be
considered the best source for bilateral state trade, but these data
suffer from a fundamental problem.  The CFS tracks shipments of goods not
the goods themselves.  For example, a rail shipment of a bushel of
corn from Eastern Nebraska to Kansas City plus the barge shipment of the
same bushel from Kansas City to New Orleans escalates the quantity
(and value) of the actual corn shipped.  This problem, of double
counting in the CFS data, is noted by \citet{AvW} who
correct the CFS interstate trade flows assuming they are exaggerated
by a factor of 2.08.  Our proposed method for generating bilateral
interstate trade imposes consistency between state-level production
and aggregate absorption. 

The state-level \emph{international} trade in the core WiNDC accounts is also
a known weakness.  The WiNDC trade data is from Census
reports of imports and exports by state, but these are actually
measured by Port of Entry.\footnote{The Census Bureau 
	disseminates export and import statistics by Port of Entry at
	the HS-6 level \citep{Census}.}  
The port from which agricultural exports exit the United States,
however, are not necessarily or even likely located in the States where
those products are produced.  For instance, corn exports departing the
ports located in the New Orleans Port District were not necessarily
grown in the State of Louisiana.  In fact, the large volume of
Louisiana exports of grains (as reported by Census and thus inferred
in the WiNDC accounts) is only explained by Louisiana's purchase of
grains through the pooled national market.  On the supply side, the
excess supply of corn in a state like Nebraska does not leave its Port
of Entry in Omaha, but rather is absorbed by the pooled national
market.  Looking at the Census data Louisiana exports a significant
quantity of corn but Nebraska, a key corn producer, does not.  A
better representation would attribute international exports (and
imports) of agricultural products to the state of production (and
absorption).  
 
Faced with the problematic Census state-trade data, the U.S.
Department of Agriculture's Economic Research Service (ERS) generates
an alternative measure of state agricultural exports based on cash
receipts.  For these estimates, the products that make up U.S.
agricultural exports are grouped to match the 24 product groups in
U.S. farm sales estimates. For each of these 24 
product groups, U.S. agricultural exports are allocated by State in
approximate proportion to the State's share of national cash receipts
for that product group. Thus, Nebraska, with 12.4 percent (\$8.9
billion) of U.S. cash receipts for corn in 2021, is estimated to have
accounted for 12.6 percent (\$2.3 billion) of U.S. corn exports that
year \citep{USDA_a, USDA_b}. In contrast, the Census Bureau's State Trade Data
indicate Nebraska's corn exports (HS-6 100590) totaled about \$609
million in 2021 \citep{Census}.

Neither of these two estimates are based on a comprehensive
assessment of the use of Nebraska's corn production, along the lines
of USDA's PSD (Production, Supply, and Distribution) Online Database
(U.S. Department of Agriculture, Foreign Agricultural Service, 2023).
Some rough estimates are commonly circulated in the industry,
however. For instance, \citet{Groskopf_Silva} estimate that about
40 percent of Nebraska's 2018 corn crop was used as feedstock for
ethanol production within the State. The \citet{NCBa}
indicates that ``about 16 percent of Nebraska's corn crop is fed to
livestock within Nebraska'' and that ``about 40 percent of the corn
grown in Nebraska is fed to livestock somewhere in the United States
or around the world.'' With respect to exports, the \citet{NCBb}
estimates that international exports account for about 
6 percent of the use of Nebraska's corn production.
          
Our purpose is to inform the key question of how much of a states
production of specific agricultural goods are exported and how much is
disbursed to each of the fifty states (plus D.C.).  We use a
structural gravity model following the theory of \citet{AvW}.  A
presentation of the full theory and its development is given by
\citet{Yotov_etal}.  There are two specific features of this theory
that we leverage in developing our estimates of interstate
agricultural trade and the flows of agricultural products between
ports and states. First is the assumption of identical and homothetic
preferences.  This provides an anchor point for calibration, where in
the absence of trade frictions expenditure shares on regionally
differentiated goods are the same across regions. The second
assumption is that trade cost are of the iceberg type.  That is, they
are paid in units of the good being shipped.  This allows us to
establish trade at normalized prices at both the frictionless anchor
and the benchmark equilibrium. 

\vspace{-8pt}
\bibliographystyle{jgea}
\bibliography{USDA_bib.bib}
\end{document}
