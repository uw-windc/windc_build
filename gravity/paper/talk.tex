\documentclass[10pt]{beamer}

\usepackage{color}
\usepackage{tikz}
\usepackage{color}
\usepackage{amsmath, amsthm, amssymb}
\usepackage{hyperref}
\usepackage{amsmath}

% Set list spacing settings:
%\usepackage{enumitem}
%\setitemize{itemsep=1ex,topsep=2ex,parsep=0pt,partopsep=0pt}

% Set margin spacing:
\setbeamersize{text margin left=8mm,text margin right=8mm} 

\title{International Trade in Agricultural Products at the
{U.S.} State Level }
\author{Thomas F. Rutherford \\
University of Wisconsin Madison}
\date{WiNDC Advisory Board Meeting \\ 10 May 2024}

\usetheme{Pittsburgh}
\usecolortheme{beaver}

\begin{document}

\begin{frame}
  \titlepage
\end{frame}

\begin{frame}{Gravity Model}


\begin{block}{What is the Gravity Model?}
A structural model that leverages both input-output accounts and
economic-geography to estimate state-level imports and exports of
agricultural products (and, eventually, other manufactured goods). 
\end{block}

\begin{block}{What does the Gravity Model do?}
 Allocates farm revenue across observed exports and
allocates imports across state-level absorption. Accounts
for economic geography under a gravity theory. 
%The more proximate farm output is to observed export nodes, for example, the greater is the export share. 
\end{block}

\begin{block}{What does this approach provide?}
 Empirically-informed estimates of the
link between farms, agricultural products, domestic absorption of
these goods, and international trade.
\vspace{.5cm}

Overcomes the attribution of state-level trade to ports of entry.

\vspace{.5cm}
 Produces a key component for assessing a given
state's exposure to trade shocks.
\end{block}

\end{frame}

\begin{frame}{Motivation}

An individual U.S. state's exposure to international markets is often
a topic of social commentary and economic analysis. Informed analysis
seems to require an estimate of a state's exports and imports.
Unfortunately, published measures of state-level trade often fall
short of accurately reflecting trade exposure. 

\end{frame}

\begin{frame}{Reported State-Level Trade Flows af Flawed}


U.S. Customs tracks international trade at specific ports of entry
(exit) rather than the ultimate destination of imports or source of
export shipments. In the official state-trade statistics from the U.S.
Census Bureau too much trade exposure is reported for states with
major sea ports or border crossings, and nearly zero exposure for
interior states producing major export commodities like wheat and
soybeans. An alternative is to simply share total U.S. exports to
states based on production shares. While transparent, this method
fails to account for trade costs over economic geography and the
co-location embeded in input-output relationships. 

\end{frame}

\begin{frame}{Problems with Census}

Census reports of imports and exports by state are measured by Port of
Entry. The port from which agricultural exports exit the United
States, however, are not necessarily or even likely located in the
States where those products are produced. For instance, corn exports
departing the ports located in the New Orleans Port District were not
necessarily grown in the State of Louisiana. In fact, the large volume
of Louisiana exports of grains (as reported by Census and thus
inferred in the WiNDC accounts) is only explained by Louisiana's
purchase of grains through the pooled national market. 

\end{frame}

\begin{frame}{ERS Estimates }
Faced with the problematic Census state-trade data, the U.S.
Department of Agriculture's Economic Research Service (ERS) generates
an alternative measure of state agricultural exports based on cash
receipts.  For these estimates, the products that make up U.S.
agricultural exports are grouped to match the 24 product groups in
U.S. farm sales estimates. For each of these 24 
product groups, U.S. agricultural exports are allocated by State in
approximate proportion to the State's share of national cash receipts
for that product group. Thus, Nebraska, with 12.4 percent (\$8.9
billion) of U.S. cash receipts for corn in 2021, is estimated to have
accounted for 12.6 percent (\$2.3 billion) of U.S. corn exports that
year. In contrast, the Census Bureau's State Trade Data
indicate Nebraska's corn exports (HS-6 100590) totaled about \$609
million in 2021.
\end{frame}


\begin{frame}{Problems with the Commodity Flow Survey}

The U.S. Department of Commerce's Bureau of Census (Census) reports in
the Commodity Flow Survey (CFS) might be considered the best source
for bilateral state trade, but these data suffer from a fundamental
problem. The CFS tracks shipments of goods not the goods themselves.
For example, a rail shipment of a bushel of corn from Eastern Nebraska
to Kansas City plus the barge shipment of the same bushel from Kansas
City to New Orleans escalates the quantity (and value) of the actual
corn shipped. This double counting in the CFS data has been noted by
Anderson and van Wincoop estimates that some flows are exagerated by
more than a factor of 2. Our proposed method for generating bilateral
interstate trade imposes consistency between state-level production
and aggregate absorption and export demand. 

\end{frame}


\begin{frame}{Structural Estimation}

We use a structural gravity model following the theory of Anderson and
van Wincoop as presented in Yotov et al. This begins with a
fundamental proposition from trade theory which posits \textit{equal
absorption shares of regionally differentiated goods in the absence of
trade frictions}.  Two specific features of this theory

\begin{enumerate}

	\item Identical and homothetic preferences. This provides an
anchor point for calibration, where in the absence of trade frictions
expenditure shares on regionally differentiated goods are the same
across regions. 

	\item Iceberg trade cost. Trade costs are paid in units of the
good being shipped. This allows us to establish trade at normalized
prices at both the frictionless anchor and the benchmark equilibrium. 

\end{enumerate}

These assumptions allow us to calibrate a commodity-specific Armington
demand system to input-output accounts that establish supply and
demand vectors for each state and each port.

\end{frame}

\begin{frame}{Interstate Trade Flows}

We then utilize the structure with trade frictions included to
establish the benchmark flows of goods from the point of production to
domestic uses and port-level exports. Similarly, we distribute
port-level imports to their use in specific states. While our primary
focus is on international trade the method yields a bilateral
interstate trade matrix for each commodity.

\end{frame}


\begin{frame}{Agricultural Products in the 43 Sector Database}


\vspace{5pt}
\begin{tabular}{ll}
\textsc{c\_b}&Sugar cane, sugar beet\\
\textsc{ctl}&Bovine cattle, sheep, goats and horses\\
\textsc{gro}&Cereal grains nec\\
\textsc{oap}&Animal products nec\\
\textsc{ocr}&Crops nec\\
\textsc{osd}&Oil seeds\\
\textsc{pdr}&Paddy rice\\
\textsc{pfb}&Plant-based fibers\\
\textsc{v\_f}&Vegetables, fruit, nuts\\
\textsc{wht}&Wheat, and\\
\textsc{wol}&Wool, silk-worm cocoons.\\
\end{tabular}
\vspace{5pt}

\end{frame}

\begin{frame}{International Trade Ports (40) }
\small
\vspace{5pt}
\begin{tabular}{lllll}
\textsc{ak\_anch}&Anchorage&&		\textsc{al\_mobi}&Mobile	     \\
\textsc{md\_balt}&Baltimore&&		\textsc{la\_newo}&New Orleans	     \\
\textsc{ma\_bost}&Boston&&		\textsc{ny\_newy}&New York City	     \\
\textsc{ny\_buff}&Buffalo&&		\textsc{az\_noga}&Nogales	     \\
\textsc{sc\_char}&Charleston&&		\textsc{va\_norf}&Norfolk	     \\
\textsc{il\_chic}&Chicago&&		\textsc{ny\_ogde}&Ogdensburg	     \\
\textsc{oh\_clev}&Cleveland&&		\textsc{nd\_pemb}&Pembina	     \\
\textsc{or\_colu}&Columbia-Snake&&	\textsc{pa\_phil}&Philadelphia	     \\
\textsc{tx\_dall}&Dallas-Fort Worth&&	\textsc{tx\_prta}&Port Arthur	     \\
\textsc{mi\_detr}&Detroit&&		\textsc{me\_port}&Portland	     \\
\textsc{mn\_dulu}&Duluth&&		\textsc{ri\_prov}&Providence	     \\
\textsc{tx\_elpa}&El Paso&&		\textsc{ca\_sand}&San Diego	     \\
\textsc{mt\_grea}&Great Falls&&		\textsc{ca\_sanf}&San Francisco	     \\
\textsc{hi\_hono}&Honolulu&&		\textsc{ga\_sava}&Savannah	     \\
\textsc{tx\_hous}&Houston-Galveston&&	\textsc{wa\_seat}&Seattle	     \\
\textsc{tx\_lare}&Laredo&&		\textsc{vt\_stal}&St. Albans	     \\
\textsc{ca\_losa}&Los Angeles&&		\textsc{mo\_stlo}&St. Louis	     \\
\textsc{fl\_miam}&Miami&&		\textsc{fl\_tamp}&Tampa		     \\
\textsc{wi\_milw}&Milwaukee&&		\textsc{dc\_wash}&Washington		\\
\textsc{mn\_minn}&Minneapolis&&		\textsc{nc\_wilm}&Wilmington.        
\end{tabular}
\vspace{5pt}

\end{frame}


\begin{frame}{}
\includegraphics[width=\textwidth]{soybeans.png}
\end{frame}

\begin{frame}{}
\includegraphics[width=\textwidth]{wheat.png}
\end{frame}
\begin{frame}{}
\includegraphics[width=\textwidth]{ex_shared_osd.png}
\end{frame}
\begin{frame}{}
\includegraphics[width=\textwidth]{ex_port_osd.png}
\end{frame}
\begin{frame}{}
\includegraphics[width=\textwidth]{ex_gravity_osd.png}
\end{frame}

\begin{frame}{Research Agenda Going Forward}

\begin{itemize}

\item Address problems with \textit{density} which arise when we
incorporate gravity estimates for bilateral trade for both countries
and states. We would like to provide a GTAP-WiNDC database which
covers most of the G20 countries, oil exporters, and rest of world, as
well as the 50 states. 

\item	Write up an overview focusing on the implications for trade in
manufactured goods and energy-intensive / trade-exposed products.

\item Leverage Census data on the bilateral composition of port-level
imports and exports to produce a GTAPWiNDC dataset which connects
state-level imports and exports to specific countries.

\end{itemize}

\end{frame}

\end{document}


